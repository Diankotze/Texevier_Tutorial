\documentclass[11pt,preprint, authoryear]{elsarticle}

\usepackage{lmodern}
%%%% My spacing
\usepackage{setspace}
\setstretch{1.2}
\DeclareMathSizes{12}{14}{10}{10}

% Wrap around which gives all figures included the [H] command, or places it "here". This can be tedious to code in Rmarkdown.
\usepackage{float}
\let\origfigure\figure
\let\endorigfigure\endfigure
\renewenvironment{figure}[1][2] {
    \expandafter\origfigure\expandafter[H]
} {
    \endorigfigure
}

\let\origtable\table
\let\endorigtable\endtable
\renewenvironment{table}[1][2] {
    \expandafter\origtable\expandafter[H]
} {
    \endorigtable
}


\usepackage{ifxetex,ifluatex}
\usepackage{fixltx2e} % provides \textsubscript
\ifnum 0\ifxetex 1\fi\ifluatex 1\fi=0 % if pdftex
  \usepackage[T1]{fontenc}
  \usepackage[utf8]{inputenc}
\else % if luatex or xelatex
  \ifxetex
    \usepackage{mathspec}
    \usepackage{xltxtra,xunicode}
  \else
    \usepackage{fontspec}
  \fi
  \defaultfontfeatures{Mapping=tex-text,Scale=MatchLowercase}
  \newcommand{\euro}{€}
\fi

\usepackage{amssymb, amsmath, amsthm, amsfonts}

\def\bibsection{\section*{References}} %%% Make "References" appear before bibliography


\usepackage[round]{natbib}
\bibliographystyle{plainnat}

\usepackage{longtable}
\usepackage[margin=2cm,bottom=2cm,top=2.5cm, includefoot]{geometry}
\usepackage{fancyhdr}
\usepackage[bottom, hang, flushmargin]{footmisc}
\usepackage{graphicx}
\numberwithin{equation}{section}
\numberwithin{figure}{section}
\numberwithin{table}{section}
\setlength{\parindent}{0cm}
\setlength{\parskip}{1.3ex plus 0.5ex minus 0.3ex}
\usepackage{textcomp}
\renewcommand{\headrulewidth}{0.2pt}
\renewcommand{\footrulewidth}{0.3pt}

\usepackage{array}
\newcolumntype{x}[1]{>{\centering\arraybackslash\hspace{0pt}}p{#1}}

%%%%  Remove the "preprint submitted to" part. Don't worry about this either, it just looks better without it:
\makeatletter
\def\ps@pprintTitle{%
  \let\@oddhead\@empty
  \let\@evenhead\@empty
  \let\@oddfoot\@empty
  \let\@evenfoot\@oddfoot
}
\makeatother

 \def\tightlist{} % This allows for subbullets!

\usepackage{hyperref}
\hypersetup{breaklinks=true,
            bookmarks=true,
            colorlinks=true,
            citecolor=blue,
            urlcolor=blue,
            linkcolor=blue,
            pdfborder={0 0 0}}


% The following packages allow huxtable to work:
\usepackage{siunitx}
\usepackage{multirow}
\usepackage{hhline}
\usepackage{calc}
\usepackage{tabularx}
\usepackage{booktabs}
\usepackage{caption}
\usepackage{colortbl}

\urlstyle{same}  % don't use monospace font for urls
\setlength{\parindent}{0pt}
\setlength{\parskip}{6pt plus 2pt minus 1pt}
\setlength{\emergencystretch}{3em}  % prevent overfull lines
\setcounter{secnumdepth}{0}

%%% Use protect on footnotes to avoid problems with footnotes in titles
\let\rmarkdownfootnote\footnote%
\def\footnote{\protect\rmarkdownfootnote}
\IfFileExists{upquote.sty}{\usepackage{upquote}}{}

%%% Include extra packages specified by user
% Insert custom packages here as follows
% \usepackage{tikz}

%%% Hard setting column skips for reports - this ensures greater consistency and control over the length settings in the document.
%% page layout
%% paragraphs
\setlength{\baselineskip}{12pt plus 0pt minus 0pt}
\setlength{\parskip}{12pt plus 0pt minus 0pt}
\setlength{\parindent}{0pt plus 0pt minus 0pt}
%% floats
\setlength{\floatsep}{12pt plus 0 pt minus 0pt}
\setlength{\textfloatsep}{20pt plus 0pt minus 0pt}
\setlength{\intextsep}{14pt plus 0pt minus 0pt}
\setlength{\dbltextfloatsep}{20pt plus 0pt minus 0pt}
\setlength{\dblfloatsep}{14pt plus 0pt minus 0pt}
%% maths
\setlength{\abovedisplayskip}{12pt plus 0pt minus 0pt}
\setlength{\belowdisplayskip}{12pt plus 0pt minus 0pt}
%% lists
\setlength{\topsep}{10pt plus 0pt minus 0pt}
\setlength{\partopsep}{3pt plus 0pt minus 0pt}
\setlength{\itemsep}{5pt plus 0pt minus 0pt}
\setlength{\labelsep}{8mm plus 0mm minus 0mm}
\setlength{\parsep}{\the\parskip}
\setlength{\listparindent}{\the\parindent}
%% verbatim
\setlength{\fboxsep}{5pt plus 0pt minus 0pt}



\begin{document}

\begin{frontmatter}  %

\title{Texevier Tutorial}

% Set to FALSE if wanting to remove title (for submission)




\author[Add1]{Dian Kotze}
\ead{kotzedian11@gmail.com}





\address[Add1]{Stellenbosch University, Stellenbosch, South Africa}

\cortext[cor]{Corresponding author: Dian Kotze}

\begin{abstract}
\small{
This is my first attempt at producing a paper using texevier. The
purpose is to teach us how to write in texevier, and include tables and
figures, similar to what we will experience when doing our project. This
is revolutionary.
}
\end{abstract}

\vspace{1cm}

\begin{keyword}
\footnotesize{
Multivariate GARCH \sep Kalman Filter \sep Copula \\ \vspace{0.3cm}
\textit{JEL classification} 
}
\end{keyword}
\vspace{0.5cm}
\end{frontmatter}



%________________________
% Header and Footers
%%%%%%%%%%%%%%%%%%%%%%%%%%%%%%%%%
\pagestyle{fancy}
\chead{}
\rhead{Financial Econometrics 871}
\lfoot{}
\rfoot{\footnotesize Page \thepage\\}
\lhead{}
%\rfoot{\footnotesize Page \thepage\ } % "e.g. Page 2"
\cfoot{}

%\setlength\headheight{30pt}
%%%%%%%%%%%%%%%%%%%%%%%%%%%%%%%%%
%________________________

\headsep 35pt % So that header does not go over title




\section{\texorpdfstring{Question 1
\label{Period 1 and 2 Moments}}{Question 1 }}\label{question-1}

The following results are the first and second moments for the periods
2006-2008, and 2010-2013. The values are very small, and with two
decimal places, some of the values appear to have mean and variance
values of zero, which is not the case. Moreover, I have interpreted that
the sample ends at the end of the previous year for both samples.

\begin{itemize}
\tightlist
\item
  2006-2008
\end{itemize}

\begin{table}[H]
\centering
\begin{tabular}{rlrr}
  \hline
 & Stocks & period1\_mean & period1\_variance \\ 
  \hline
1 & JSE.ABSP.Close & -0.00 & 0.00 \\ 
  2 & JSE.BVT.Close & 0.00 & 0.00 \\ 
  3 & JSE.FSR.Close & 0.00 & 0.00 \\ 
  4 & JSE.NBKP.Close & -0.00 & 0.00 \\ 
  5 & JSE.RMH.Close & 0.00 & 0.00 \\ 
  6 & JSE.SBK.Close & 0.00 & 0.00 \\ 
  7 & JSE.SLM.Close & 0.00 & 0.00 \\ 
   \hline
\end{tabular}
\caption{Moments of the stocks for 2006-2008 \label{tab1}} 
\end{table}

\begin{itemize}
\tightlist
\item
  2010-2013

  \begin{table}[H]
  \centering
  \begin{tabular}{rlrr}
    \hline
   & Stocks & period1\_mean & period1\_variance \\ 
    \hline
  1 & JSE.ABSP.Close & 0.00 & 0.00 \\ 
    2 & JSE.BVT.Close & 0.00 & 0.00 \\ 
    3 & JSE.FSR.Close & 0.00 & 0.00 \\ 
    4 & JSE.NBKP.Close & 0.00 & 0.00 \\ 
    5 & JSE.RMH.Close & 0.00 & 0.00 \\ 
    6 & JSE.SBK.Close & 0.00 & 0.00 \\ 
    7 & JSE.SLM.Close & 0.00 & 0.00 \\ 
     \hline
  \end{tabular}
  \caption{Moments of the stocks for 2010-2013 \label{tab1}} 
  \end{table}
\end{itemize}

The most notable differences between the two periods is the negative
mean values in period 1 (2006-2008) for ABSP and NBKP. Additionally, the
variance between the two periods differs substantially for every stock
except NBKP and RMH. ABSP has the highest variance in returns across
both periods, suggesting that it is the most volatile of the stocks
(Katzke \protect\hyperlink{ref-Texevier}{2017} \& Tsay
(\protect\hyperlink{ref-Tsay1989}{1989})).

\section{\texorpdfstring{Question 2: Calculating the full sample
correlations between the stocks
\label{Question 2: Calculate the full sample correlations between the stocks}}{Question 2: Calculating the full sample correlations between the stocks }}\label{question-2-calculating-the-full-sample-correlations-between-the-stocks}

\begin{table}[H]
\centering
\scalebox{0.7}{
\begin{tabular}{rrrrrrrr}
  \hline
 & JSE.ABSP.Close & JSE.BVT.Close & JSE.FSR.Close & JSE.NBKP.Close & JSE.RMH.Close & JSE.SBK.Close & JSE.SLM.Close \\ 
  \hline
JSE.ABSP.Close & 1.00 & -0.42 & -0.44 & 0.92 & -0.44 & -0.48 & -0.48 \\ 
  JSE.BVT.Close & -0.42 & 1.00 & 0.95 & -0.41 & 0.93 & 0.90 & 0.98 \\ 
  JSE.FSR.Close & -0.44 & 0.95 & 1.00 & -0.43 & 0.98 & 0.93 & 0.97 \\ 
  JSE.NBKP.Close & 0.92 & -0.41 & -0.43 & 1.00 & -0.43 & -0.45 & -0.48 \\ 
  JSE.RMH.Close & -0.44 & 0.93 & 0.98 & -0.43 & 1.00 & 0.94 & 0.94 \\ 
  JSE.SBK.Close & -0.48 & 0.90 & 0.93 & -0.45 & 0.94 & 1.00 & 0.91 \\ 
  JSE.SLM.Close & -0.48 & 0.98 & 0.97 & -0.48 & 0.94 & 0.91 & 1.00 \\ 
   \hline
\end{tabular}
}
\caption{Unconditional Correlations Between Stocks \label{t}} 
\end{table}

\section{\texorpdfstring{Question 3: Plot the univariate GARCH ht
processes for each of the series
\label{Question 3: Plot the univariate GARCH ht processes for each of the series}}{Question 3: Plot the univariate GARCH ht processes for each of the series }}\label{question-3-plot-the-univariate-garch-ht-processes-for-each-of-the-series}

Note that ``Data\_wide''" is created which has the log returns of all
the stocks. This data has been spread, after the log returns were
calclulated for each stocks from the dailydata.

\begin{verbatim}
## 
## please wait...calculating quantiles...
\end{verbatim}

\begin{figure}[H]

{\centering \includegraphics{Dian_Tutorial_files/figure-latex/figure1-1} 

}

\caption{GARCH Plot \label{lit}}\label{fig:figure1}
\end{figure}

\section{\texorpdfstring{Question 4: Plot the cumulative returns series
of a portfolio that is equally weighted to each of the stocks -
reweighted each year on the last day of June
\label{Question 4}}{Question 4: Plot the cumulative returns series of a portfolio that is equally weighted to each of the stocks - reweighted each year on the last day of June }}\label{question-4-plot-the-cumulative-returns-series-of-a-portfolio-that-is-equally-weighted-to-each-of-the-stocks---reweighted-each-year-on-the-last-day-of-june}

\begin{figure}[H]

{\centering \includegraphics{Dian_Tutorial_files/figure-latex/figure2-1} 

}

\caption{Cumulative returns plot- equally weighted \label{lit}}\label{fig:figure2}
\end{figure}

Unfortunately I was unable to reweight the portfolio on the last day of
June each year. I tried using what was provided in practical 5, and I
have the intuition behind it. However, there appears to be an error in
my code (clearly I am missing something).

As such, this brings an end to my first paper produced in texevier.
Excting times ahead!

\newpage

\section*{References}\label{references}
\addcontentsline{toc}{section}{References}

\hypertarget{refs}{}
\hypertarget{ref-Texevier}{}
Katzke, N.F. 2017. \emph{Texevier: Package to Create Elsevier Templates
for Rmarkdown}. Stellenbosch, South Africa: Bureau for Economic
Research.

\hypertarget{ref-Tsay1989}{}
Tsay, Ruey S. 1989. ``Testing and Modeling Threshold Autoregressive
Processes.'' \emph{Journal of the American Statistical Association} 84
(405). Taylor \& Francis Group: 231--40.

% Force include bibliography in my chosen format:

\bibliographystyle{Tex/Texevier}
\bibliography{Tex/ref}





\end{document}
